\documentclass{article}
\usepackage[utf8]{inputenc}
\title{Estructures Algebraiques\\Grups}
\author{Sistach Reinoso, Arnau}

% nomes crec que serveix per la ela geminada (on sense el corrector dubto mai fer-ho anar 183 ·)
\usepackage[catalan]{babel}

% perque quedi mes clar
\usepackage{color}


\usepackage{mathtools}

% faig anar equation*
\usepackage{amsmath}

\usepackage{amssymb}
\newcommand{\N}{\mathbb{N}}
\newcommand{\Z}{\mathbb{Z}}
\newcommand{\Q}{\mathbb{Q}}
\newcommand{\R}{\mathbb{R}}
\newcommand{\C}{\mathbb{C}}

\newcommand{\prodEscalar}[2]{\langle#1, #2\rangle}



% like kindle

\usepackage{microtype}     % microtypography, reduces hyphenation

\usepackage[papersize={3.6in,4.8in},hmargin=0.1in,vmargin={0.1in,0.1in}]{geometry}  % page geometry

\usepackage{fancyhdr}   % headers and footers
\pagestyle{fancy}
\fancyhead{}            % clear page header
\fancyfoot{}            % clear page footer

\setlength{\abovecaptionskip}{2pt} % space above captions 
\setlength{\belowcaptionskip}{0pt} % space below captions
\setlength{\textfloatsep}{2pt}     % space between last top float or first bottom float and the text
\setlength{\floatsep}{2pt}         % space left between floats
\setlength{\intextsep}{2pt}        % space left on top and bottom of an in-text float


% https://atlas.mat.ub.edu/personals/crespo/ApuntsEstructuresAlgebraiques.pdf
\begin{document}
\maketitle
\tableofcontents

\section{L'espai $\R^2$}
\begin{itemize}
\item $x - (x_1, x_2, \dots, x_n)$, $x_i$ són les coordenades
\item $\R^2$ plà euclidià
\item $\R^3$ espai euclidià
\end{itemize}

\subsection{Demostració}
\begin{itemize}
\item Apliquem el teorema del cosinua al triangle
\subitem $\|x-y\|^2 = \|x\|^2 - 2 \|x\|\|y\| \cos{\theta}$
\item $\|x-y\|^2 = (x - y) (x - y) = \|x\|^2 + \|y\|^2 - 2x\cdot y$
\item $\|x\|^2 - 2 \|x\|\|y\| \cos{\theta} = \|x\|^2 + \|y\|^2 - 2x\cdot y$
\end{itemize}

\section{Definicions}
\begin{itemize}
\item Producte escalar (euclidià)
\begin{itemize}
	\item $ \forall x, y \in \R^n\quad x\cdot y = \prodEscalar{x}{y} = \sum_{i=1}^n x_iy_i \in \R$
	\item $ \forall \theta \in [0, \pi] \quad \prodEscalar{x}{y} = \|x\|\|y\| \cos{\theta}$
\end{itemize}
\item Norma euclidiana
	\begin{itemize}
	\item $\forall x \in \R^n \quad \|x\| = \prodEscalar{x}{x}^{\frac{1}{2}} = \sqrt{\sum x_i^2}$
	\end{itemize}
\item unitari
	\begin{itemize}
	\item $u$ si $\|u\| = 1$
	\item normalitzar un vectoro
		\subitem $\forall x \in \R^n\setminus \{0\} \to u = \frac{x}{\|x\|}$
	\end{itemize}

\item Ortogonals
	\begin{itemize}
	\item $\prodEscalar{x}{y} = 0 \Rightarrow x \perp y$
	\item Projecció O. d'un vector sobre un altre $\neq \emptyset$
		\begin{itemize}
		\item
		$
		\begin{rcases}
		v = v_1 + v_2\\ v_1 = \lambda u \\ v_2 \cdot u = 0
		\end{rcases} \Leftrightarrow 0 = (v - v_1)u 
		$
		\item $0 = (v - \lambda u) u = v\cdot u - \lambda u\cdot u = v\cdot u - \lambda \|u\|^2$
		\item $\Leftrightarrow \lambda = \frac{v\cdot u}{\|u\|^2}$
		\end{itemize}
	\end{itemize}
\end{itemize}
\section{Propietats}
\begin{itemize}
\item Producte Esclar
	\begin{itemize}
	\item $\prodEscalar{x}{x} \ge 0$ i $\prodEscalar{x}{x} = 0 \Leftrightarrow x = \vec0$
	\item $\prodEscalar{x}{y} = \prodEscalar{y}{x}$
	\item $\forall \alpha, \beta\quad (\alpha x + \beta y)\cdot z = \alpha(x\cdot z) + \beta(y\cdot z)$
	\end{itemize}
\item Norma
	\begin{itemize}
	\item $\|x\| \ge 0$
	\item $\|x\| = 0 \Leftrightarrow x = \vec 0$
	\item $\|\lambda x\| = |\lambda|\|x\|$ Homogeneïtat
	\item $\|x + y\| \le \|x\| + \|y\|$ Desigualtat triangular
		\begin{itemize}
		\item Demostració
		\item $\|x + y\|^2 = (x + y)(x + y) =$
		\item $\|x\|^2 + 2x\cdot y + \|y\|^2 \overset{\text{Cauchy-Schwarz}}{\le}$
		\item $\|x\|^2 + 2\|x\|\|y\| + \|y\|^2 = $
		\item $(\|x\| + \|y\|)^2$
		\item Conseqüència triangular
		\item $|\|x\|-\|y\|| \le \|x - y\|$
		\item $-\|x-y\| \le \|x\| - \|y\| \le \|x - y\|$
			\subitem $\|y\| \le \|x - y\| + \|x\|$
			\subitem $\|x\| \le \|x - y\| + \|y\|$
		\end{itemize}
	\end{itemize}
\end{itemize}

\section{Coses fortes}
\begin{itemize}
\item Desigualtat de Chauchy-Schwarz
	\subitem $|x\cdot y| \le \|x\|\|y\|$
\end{itemize}


\end{document}
